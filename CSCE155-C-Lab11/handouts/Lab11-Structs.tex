\documentclass[12pt]{scrartcl}


\setlength{\parindent}{0pt}
\setlength{\parskip}{.25cm}

\usepackage{graphicx}

\usepackage{xcolor}

\definecolor{darkred}{rgb}{0.5,0,0}
\definecolor{darkgreen}{rgb}{0,0.5,0}
\usepackage{hyperref}
\hypersetup{
  letterpaper,
  colorlinks,
  linkcolor=red,
  citecolor=darkgreen,
  menucolor=darkred,
  urlcolor=blue,
  pdfpagemode=none,
  pdftitle={Introduction To Git},
  pdfauthor={Christopher M. Bourke},
  pdfcreator={$ $Id: cv-us.tex,v 1.28 2009/01/01 00:00:00 cbourke Exp $ $},
  pdfsubject={PhD Thesis},
  pdfkeywords={}
}

\definecolor{MyDarkBlue}{rgb}{0,0.08,0.45}
\definecolor{MyDarkRed}{rgb}{0.45,0.08,0}
\definecolor{MyDarkGreen}{rgb}{0.08,0.45,0.08}

\definecolor{mintedBackground}{rgb}{0.95,0.95,0.95}
\definecolor{mintedInlineBackground}{rgb}{.90,.90,1}

%\usepackage{newfloat}
\usepackage[newfloat=true]{minted}
\setminted{mathescape,
               linenos,
               autogobble,
               frame=none,
               framesep=2mm,
               framerule=0.4pt,
               %label=foo,
               xleftmargin=2em,
               xrightmargin=0em,
               startinline=true,  %PHP only, allow it to omit the PHP Tags *** with this option, variables using dollar sign in comments are treated as latex math
               numbersep=10pt, %gap between line numbers and start of line
               style=default, %syntax highlighting style, default is "default"
               			    %gallery: http://help.farbox.com/pygments.html
			    	    %list available: pygmentize -L styles
               bgcolor=mintedBackground} %prevents breaking across pages
               
\setmintedinline{bgcolor={mintedBackground}}
\setminted[text]{bgcolor={mintedBackground},linenos=false,autogobble,xleftmargin=1em}
%\setminted[php]{bgcolor=mintedBackgroundPHP} %startinline=True}
\SetupFloatingEnvironment{listing}{name=Code Sample}
\SetupFloatingEnvironment{listing}{listname=List of Code Samples}


\title{CSCE 155 - C}
\subtitle{Lab 11 - Structs}
\author{Dr.\ Chris Bourke}
\date{~}

\begin{document}

\maketitle

\section*{Prior to Lab}

Before attending this lab:
\begin{enumerate}
  \item Read and familiarize yourself with this handout.
  \item Review the following free textbook resources:
	\begin{itemize}
  	  \item \url{http://en.wikibooks.org/wiki/C_Programming/Complex_types#Structs}
	  \item \url{http://www.cs.cf.ac.uk/Dave/C/node9.html}
	\end{itemize}
  \item For additional Information on some advanced topics:
  \begin{itemize}
    \item RSS: \url{http://en.wikipedia.org/wiki/RSS}
    \item HTTP Protocol: \url{http://en.wikipedia.org/wiki/HTTP}
    \item C Sockets: \url{http://www.linuxhowtos.org/C_C++/socket.htm}
  \end{itemize}
\end{enumerate}

\section*{Peer Programming Pair-Up}

To encourage collaboration and a team environment, labs will be
structured in a \emph{pair programming} setup.  At the start of
each lab, you will be randomly paired up with another student 
(conflicts such as absences will be dealt with by the lab instructor).
One of you will be designated the \emph{driver} and the other
the \emph{navigator}.  

The navigator will be responsible for reading the instructions and
telling the driver what to do next.  The driver will be in charge of the
keyboard and workstation.  Both driver and navigator are responsible
for suggesting fixes and solutions together.  Neither the navigator
nor the driver is ``in charge.''  Beyond your immediate pairing, you
are encouraged to help and interact and with other pairs in the lab.

Each week you should alternate: if you were a driver last week, 
be a navigator next, etc.  Resolve any issues (you were both drivers
last week) within your pair.  Ask the lab instructor to resolve issues
only when you cannot come to a consensus.  

Because of the peer programming setup of labs, it is absolutely 
essential that you complete any pre-lab activities and familiarize
yourself with the handouts prior to coming to lab.  Failure to do
so will negatively impact your ability to collaborate and work with 
others which may mean that you will not be able to complete the
lab.  

\section{Lab Objectives \& Topics}
At the end of this lab you should be familiar with the following
\begin{itemize}
  \item Be familiar with the concepts of encapsulation \& modularity
  \item Understand how to design, declare, and use C structures 
  	(both by reference and by value)
  \item Have some exposure to advanced topics such as sockets, 
	the HTTP protocol, and XML processing
\end{itemize}

\section{Background}

An RSS feed (RDF Site Summary or ``Really Simple Syndication'') 
is a format used to publish frequently updated works.  RSS enabled 
clients can subscribe to RSS feeds and update a user as to new or 
relevant news items.  RSS feeds are most commonly formatted using 
XML (Extensible Markup Language) that use XML tags to indicate 
what the data represents (the title of the article, a short description, 
etc.).  Clients ``read'' an RSS feed by making a connection to a server 
using the HyperText Transfer Protocol (HTTP).

For example, UNL has an RSS news feed available at \url{http://newsroom.unl.edu/releases/?format=xml} 
which serves XML data that looks something like the following:

\begin{minted}[breaklines]{xml}
<rss xmlns:media="http://search.yahoo.com/mrss/" xmlns:atom="http://www.w3.org/2005/Atom" version="2.0">
<channel>
  <title>UNL News Releases</title>
  <link>http://newsroom.unl.edu/releases/</link>
  <description>News from the University of Nebraska-Lincoln</description>
  <language>en-us</language>
  <copyright>Copyright 2012 University of Nebraska-Lincoln</copyright>
  <image>
    <title>UNL News Releases</title>
    <url>http://www.unl.edu/favicon.ico</url>
    <link>http://www.unl.edu/</link>
  </image>
  <item>
    <title>Guerrilla Girls on Tour perform 'Feminists are Funny' Monday at Sheldon</title>
    <link>http://newsroom.unl.edu/releases/2012/03/09/Guerrilla</link>
    <description>The Guerrilla Girls on Tour, an internationally acclaimed anonymous theater collective, will perform "Feminists are Funny" at the University of Nebraska-Lincoln's Sheldon Museum of Art, 12th and R streets, at 7 p.m. March 12. The 70-minute play is an...
    </description>
    <pubDate>Fri, 09 Mar 2012 02:00:00 -0600</pubDate>
  </item>
  ...
  </items>
</channel>
\end{minted}

\subsection*{Structures in C}

An entity may be composed of several different pieces of data.  
A person for example may have a first and last name (strings), 
an age (integer), a birthdate (some date/time representation), 
etc.  Its much easier and more natural to group each of these 
pieces of data into one entity.  This is a concept known as 
\emph{encapsulation}--a mechanism by which data can be 
grouped together to define an entity.

The C programming language provides a mechanism to achieve 
encapsulation using structures.  Structures are user defined types 
that have one or more data fields--variables which have a type 
and a name.  To access the member fields of a structure you can 
use the dot operator; example: \mintinline{c}{student.firstName}.  
However, when you have a reference to a structure, you need to 
use the arrow operator: \mintinline{c}{student->firstName}.

\subsection*{RSS Client Background}

You have been provided with an incomplete RSS client written in C.  
The client works as follows: based on a URL (which consists of a 
host and a resource on that host), it opens a socket connection and 
sends a GET request to the server.  The server responds with a 
stream of data that the client reads into a buffer.  This data stream 
can, in general, be any type of data, but we're expecting an RSS 
feed--a stream of plain text XML-formatted data conforming to the 
RSS standard.  Since the data is plain text, it is stored in a 
character array which is then handed off to another function 
to parse the XML document.  This is done using a library (libxml2, 
an XML parser and toolkit for Gnome).

\section{Activities}

Clone the project code for this lab from GitHub by using the following URL:
\url{https://github.com/cbourke/CSCE155-C-Lab11}.

\subsection{A Student Structure}

This activity will familiarize you with a completed program in which 
a C structure has been declared to represent a student.  Several 
functions have been implemented to assist in the construction and 
printing of this structure.  In particular there are two ``factory'' 
functions that can be used to help in the construction of an individual 
structure.  The factory function takes the appropriate parameters, 
allocates memory to hold a new student and sets each field of the 
student appropriately.  In the case of strings, malloc is used to create 
enough space and the string is then copied.  The birthdate is handled 
specially: it parses the date string and creates a \mintinline{c}{struct tm} 
which is a \emph{time} structure defined by the time library 
(\mintinline{text}{time.h}).  Ultimately, a pointer to the new student 
structure is returned.  There is also a print function that takes a 
student (by reference) and prints it out to the standard output.


\subsubsection*{Instructions}

\begin{enumerate}
  \item Examine the syntax and program structure and observe how 
	each factory function works as well as how they are used.
  \item Change the values in the \mintinline{c}{main} function to your 
	name, NUID, and birth date.
  \item Compile and run the program.  Refer back to this program in 
	the next activity as needed.
\end{enumerate}
	
\subsection{Completing the RSS Client}

In this activity, you will complete the RSS Client that connects to a UNL 
RSS feed, processes the XML data and outputs the results to the 
standard output.  Most of the client has been completed for you.  
You just need to complete the design and implementation of a C 
structure that models the essential parts of an RSS item.  Your 
structure will need to support an RSS item's title, link, description, 
and publication date.  

As a first attempt you can treat the publication date as a string, 
but ultimately you should use a \mintinline{c}{tm} structure similar 
to the \mintinline{c}{Student} example.  This structure has several 
fields that give you access to the year, month, date, hour, minute, 
etc.  For full documentation, see the following: \url{http://pubs.opengroup.org/onlinepubs/7908799/xsh/time.h.html}

\subsubsection*{Instructions}

\begin{enumerate}
  \item We have provided 3 different RSS feeds to work with in the
    program.  You can switch between them using command line arguments.
  \item Design and implement the RSS structure 
  \item Complete the other functions
  \begin{itemize}
    \item \mintinline{c}{createEmptyRss()} - this function should create an 
  	``empty'' RSS structure
    \item \mintinline{c}{createRss()} - this function should construct an RSS 
	structure based on the input parameters.  For the date, use a string
	representation (RSS and Atom feeds can use a variety of formats and
	there is no easy way to automatically parse them).
	\item \mintinline{c}{printRss()} - this function should print the given RSS 
	to the standard output in a human readable format (the details are up to you).  
  \end{itemize}	
  \item You will need to use the makefile to compile your program as it 
	contains the flags necessary to import the proper XML libraries.  
	Recall that you can execute the makefile by typing the \mintinline{text}{make} 
	command.
\end{enumerate}

\section{Advanced Activity (Optional)}

\begin{enumerate}
  \item You will notice that internally, we have provided support for both
  RSS 2.0 feeds and Atom 1.0 feeds (Reddit uses Atom for example).  
  Examine the files where we have defined these feeds.  Find an RSS or Atom
  feed that is of interest to you and integrate it as an option for this program.
  Be sure to change the main function so that it can be used.
  
  \item Improve the program further by modifying the \mintinline{c}{parseRSS_XML} 
	function.  Currently, this function parses the XML but limits the number of 
	RSS structures to a maximum of 100.  Write another function to count the 
	number of items in the XML and use it to instead dynamically allocate an 
	RSS array of a size exactly equal to the number of item elements in the XML file.
\end{enumerate}

\end{document}
